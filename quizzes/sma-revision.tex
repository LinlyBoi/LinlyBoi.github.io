\documentclass[theme=sleek, randomorder, hidesidemenu]{webquiz}
\usepackage{graphicx}
\graphicspath{ {./sma-revision/} }
\DeclareGraphicsExtensions{.png, .jpg}
\title{SMA Revision}
\begin{document}

\begin{question}
  Google uses PageRank to \ldots links
  \begin{choice}[columns=2]
    \incorrect add
    \incorrect alter
    \correct analyze
    \incorrect remove
  \end{choice}
\end{question}

\begin{question}
  Youtube and Flickr are examples of \ldots service
  \begin{choice}[columns=2]
    \incorrect blogging
    \incorrect microblogging
    \correct content community
    \incorrect social networking
  \end{choice}
\end{question}

\begin{question}
  What is the output of the line of code shown below?\\
  \verb|word_stemmer.stem('running')|
  \begin{choice}
    \correct run
    \incorrect running
    \incorrect runn
    \incorrect None of the mentioned
  \end{choice}
\end{question}

\begin{question}
Search engine which takes input from a user and simultaneously send out
queries to third party search engines for results, is \ldots
\begin{choice}
  \incorrect Advanced search engine
  \incorrect Boolean search engine
  \correct Meta search engine
  \incorrect Multi search engine
\end{choice}
\end{question}

\begin{question}
  If a user returns to your site, they will count as a user but not a new user.
  \begin{choice}
    \correct True
    \incorrect False
  \end{choice}
\end{question}

\begin{question}
News bots are categorized as social bots
\begin{choice}[columns=2]
  \incorrect True
  \correct False
\end{choice}
\end{question}

\begin{question}
  Today, more people are getting their news instantly from online news
websites, social media platforms, blogs, among others. News bots has
a……………. imitation of human behavior and a……………. intention.

\begin{choice}[columns=2]
  \incorrect low, malicious
  \incorrect high, malicious
  \correct low, benign
  \incorrect high, benign
\end{choice}
\end{question}

\begin{question}
  Select the second name of the Web crawler
  \begin{choice}[columns=2]
    \correct Web spider
    \incorrect Search optimizer
    \incorrect Web manager
    \incorrect Link directory
  \end{choice}
\end{question}

\begin{question}
  Which of the following types applies to the Web crawler?
  \begin{choice}[columns=2]
    \incorrect Repetitive
    \incorrect Reactive
    \correct Innocent
    \incorrect Repetitive and Reactive
  \end{choice}
\end{question}

\begin{question}
  In \ldots data is analytical friendly
  \begin{choice}[columns=2]
    \incorrect Social Data Analytics
    \correct Business Analytics
    \incorrect Search Engine Analytics
    \incorrect Social Data and Search Engine
  \end{choice}
\end{question}

\begin{question}
  What is the main goal of an influencer bot?
  \begin{choice}[columns=2]
    \incorrect To build a loyal following
    \correct To inflate popularity (person or brand)
    \incorrect To provide customer service and answer user inquiries
    \incorrect To share high-quality content
  \end{choice}
\end{question}

\begin{question}
  Influencer bots can inflate a brand's
  \begin{choice}[columns=2]
    \incorrect Brand value
    \correct Engagement metrics (likes, comments)
    \incorrect Customer satisfaction
    \incorrect All of the mentioned
  \end{choice}
\end{question}

\begin{question}
 Over time, influencer  bots can lead to
 \begin{choice}[columns=2]
   \incorrect Decrease brand communities built on genuine interactions
   \incorrect Less trust in influencers
   \incorrect They skew campaign results and waste resources
   \correct All the mentioned
 \end{choice}
\end{question}

\begin{question}
  Brands can protect themselves from bots by
  \begin{choice}[columns=2]
    \incorrect Focusing solely on influencers with the highest follower count
    \correct Utilizing tools that can analyze activity for signs of bot activity
    \incorrect Encouraging the use of influencer bots to maximize reach
    \incorrect Prioritizing influencers with the most visually appealing content
  \end{choice}
\end{question}

\begin{question}
  If site A has a 47\% bounce rate, and another site B that competes with it has a
  60\% bounce rate, then ……… can be considered a better page to show in
  response to a user’s search query
  \begin{choice}[columns=2]
    \correct Site A
    \incorrect Site B
    \incorrect both A and B
    \incorrect Neither
  \end{choice}
\end{question}

\begin{question}
  Students-Teachers relationship is better represented using…………network
  \begin{choice}[columns=2]
    \incorrect Undirected
    \incorrect One-mode
    \correct Two-mode
    \incorrect Multi-mode
  \end{choice}
\end{question}

\begin{question}
  For a certain given directed graph $G = (N, E)$, where $N = 5$, what is the maximum number of edges between the given nodes?
  \begin{choice}
    \incorrect 10
    \correct 20
    \incorrect 5
    \incorrect 25
  \end{choice}
\end{question}

\begin{question}
  For $G = (N, E)$, which Python Function could be used to get the count of $N$?
  \begin{choice}[columns=2]
    \incorrect G.size()
    \incorrect nx.degree(G)
    \correct G.order()
    \incorrect nx.Graph()
  \end{choice}
\end{question}

\begin{question}
  Which Python Function could be used to get the degree centrality of a given node?
  \begin{choice}[columns=2]
    \incorrect G.size()
    \correct nx.degree(G, ``node'')
    \incorrect G.order()
    \incorrect nx.Graph()
  \end{choice}
\end{question}

\begin{question}
  If Bob visits a.com and clicks a link to your site, he starts a session attributed
to a………… from a.com
\begin{choice}[columns=2]
  \incorrect indirect link
  \correct referral
  \incorrect social embedding
  \incorrect All the mentioned
\end{choice}
\end{question}

\begin{question}
  For the tweet given below, which type of sentiment scoring is the most suitable
to be applied
  \begin{center}
    \includegraphics[height=50mm, width=50mm]{tweet.png}
  \end{center}
  \begin{choice}[columns=2]
    \correct Emotion detection
    \incorrect Polarity Classification
    \incorrect Graded sentiment
    \incorrect None of the mentioned
  \end{choice}
\end{question}

\begin{question}
  The given tweet faces a \ldots challenge while being prepared for analysis
  \begin{center}
    \includegraphics[height=50mm, width=50mm]{tweet.png}
  \end{center}
  \begin{choice}[columns=2]
    \incorrect Unstructuredness
    \incorrect Diversity
    \incorrect Velocity
    \correct Unstructuredness and Diversity
  \end{choice}
\end{question}

\begin{question}
    How many opinion elements are obviously included in the given tweet
  \begin{center}
    \includegraphics[height=50mm, width=50mm]{tweet.png}
  \end{center}
  \begin{choice}[columns=2]
    \correct 3
    \incorrect 1
    \incorrect 5
    \incorrect 2
  \end{choice}

\end{question}

\begin{question}
  Which pre-processing step is essential for handling case sensitivity in text analysis?
  \begin{choice}[columns=2]
    \incorrect Lemmatization
    \incorrect Stop word removal
    \incorrect Tokenization
    \correct Lowercasing
  \end{choice}
\end{question}

\begin{question}
lower()method in Question 20 is applicable to the given text in the given tweet
  \begin{center}
    \includegraphics[height=50mm, width=50mm]{tweet.png}
  \end{center}
  \begin{choice}[columns=2]
    \incorrect True
    \correct False
  \end{choice}
\end{question}

\begin{question}
  Justin Bieber is often mentioned in influencer marketing discussions, but some consider him a ``Fake Influencer'' along with other celebrities. An analysis revealed that only 63\% of his 168.7 million followers are genuine fans, with an engagement rate of 0.37\%. What is the most likely explanation for the remaining 37\% of followers?

  \begin{choice}[columns=2]
    \incorrect Fake accounts
    \incorrect Social bots
    \incorrect Influencer Bots
    \correct All of the mentioned
  \end{choice}
\end{question}

\begin{question}
  \ldots is the act of incorporating social media content (e.g., a link, video, or presentation) into a website or blog
  \begin{choice}[columns=2]
    \correct embedding
    \incorrect clicking
    \incorrect mentioning
    \incorrect sharing
  \end{choice}
\end{question}

\begin{question}
  If one shared post was: ``\#PingIt guys this is disappointing :('', how many keywords are exposed to the lemmatization process? (treat the :( as an emoji)
  \answer[number]{2}
\end{question}

\begin{question}
  How many replacements could be made while preprocessing the post given:
  ``\#PingIt guys this is disappointing :('' (treat the :( as an emoji)
  \answer[number]{1}
\end{question}

\begin{question}
  Reviewers of the PingIt application, all over the world, constructed \ldots network type
  \begin{choice}[columns=2]
    \incorrect weighted
    \incorrect unweighted
    \incorrect explicit
    \correct implicit
  \end{choice}
\end{question}

\begin{question}
Knowing that PingIt is popular on social media doesn’t tell you whether that’s
a good thing or a bad thing. Text analytics tells you
\begin{choice}
  \correct True
  \incorrect False
\end{choice}
\end{question}

\end{document}
