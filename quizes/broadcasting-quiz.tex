\documentclass[theme=sleek, randomorder, hidesidemenu]{webquiz}
\title{Digital Broadcasting}
\begin{document}

\begin{question}
  an example of narrowcasting is
  \begin{choice}
    \incorrect Tv channel targetting students
    \correct School network broadcast
    \incorrect Government broadcast
    \incorrect Internet broadcast
  \end{choice}

\end{question}

\begin{question}
  For something to be considered digital broadcasting it has to be over the internet
  \begin{choice}
    \incorrect True \feedback can be any digital network
    \correct False
  \end{choice}

\end{question}

\begin{question}
  Satellite is not a type of digital broadcasting
  \begin{choice}
    \incorrect True
    \correct False \feedback it occurs over a digital network
  \end{choice}

\end{question}

\begin{question}
  We can have multiple radio channels using the same frequency in the same region by using
  \begin{choice}
    \correct AM
    \incorrect FM \feedback AM has a shorter range
  \end{choice}
\end{question}

\begin{question}
  Audio broadcasting is another name for radio broadcasting
  \begin{choice}
    \incorrect True
    \correct False
    \feedback Audio broadcasting is transmitting digital audio, while radio is analogue
  \end{choice}
\end{question}

\begin{question}
  FM Radio is based on
  \begin{choice}
    \correct Frequency
    \incorrect Amplitude
    \incorrect Wavelength
    \incorrect Periodic Time
  \end{choice}
\end{question}

\begin{question}
  AM Radio is based on
  \begin{choice}
    \incorrect Frequency
    \correct Amplitude
    \incorrect Wavelength
    \incorrect Periodic Time
  \end{choice}
\end{question}

\begin{question}
  Which one has a longer range? FM or AM?
  \answer[lower] FM
\end{question}

\begin{question}
  One of the main key points when writing a script is
  \begin{choice}
    \correct Saying things with the right tone to portray the correct meaning
    \incorrect Have some bias to stand out as your own person
    \feedback no bias at all
    \incorrect Make it obvious when a section or chapter is ending to inform the listener
    \feedback seemlessly transition is better
    \incorrect Do minimal research to leave room for discussion and guest to introduce themselves in their own way
    \feedback It is important to do your research properly
  \end{choice}
\end{question}

\begin{question}
    Maintaining ethics and objectivity is crucial when writing a good script
  \begin{choice}
    \correct True
    \incorrect False
  \end{choice}
\end{question}

\begin{question}
    Digital broadcasting is the distribution of video or audio content to a large audience via electronic mass communication
  \begin{choice}
    \correct True
    \incorrect False
  \end{choice}
\end{question}

\begin{question}
  Television is a form of digital broadcasting
  \begin{choice}
    \correct True
    \incorrect False
  \end{choice}
\end{question}

\begin{question}
  Television is a form of digital broadcasting
  \begin{choice}
    \correct True
    \incorrect False
  \end{choice}
\end{question}

\begin{question}
  Public broadcasting is when broadcasts are available to everyone free of charge and can be monetised with ads or sponsors
  \begin{choice}
    \incorrect True
    \correct False
  \end{choice}
\end{question}

\begin{question}
  Using technical terminology can be good to showcase to the audience that you understand the topic and are an expert in the field
  \begin{choice}
    \incorrect True \feedback keep words simple so everyone understands
    \correct False
  \end{choice}
\end{question}

\begin{question}
  A Transmitter is crucial for a digital broadcast
  \begin{choice}
    \correct True
    \incorrect False
    \feedback it doesn't have to be a physical device can be software
  \end{choice}
\end{question}

\begin{question}
  Multiplexing is sending out multiple signals together and recieved as one but can't be split at the destination
  \begin{choice}
    \incorrect True
    \correct False \feedback it is split at the destination
  \end{choice}

\end{question}

\begin{question}
  From the radio principles
  \begin{choice}[multiple]
    \correct Showcase all sides of an argument regardless of conflict type
    \incorrect Express your opinion but ensure it is not stated as fact
    \incorrect Words should be as written to portray the correct meaning every time
    \correct Use easy simple short sentences

  \end{choice}
\end{question}

\begin{question}
  a program's structure doesn't include how it would be monetised

  \begin{choice}
    \incorrect True
    \correct False
  \end{choice}

\end{question}

\begin{question}
  Staff and resources are part of the radio program structure

  \begin{choice}
    \correct True
    \incorrect False
  \end{choice}

\end{question}

\begin{question}
  It is important to know our target audience and define them
  \begin{choice}
    \correct True
    \incorrect False

  \end{choice}

\end{question}

\begin{question}
  From the platforms for listening to broadcasts
  \begin{choice}
    \incorrect TV
    \incorrect Computer
    \incorrect Radio
    \incorrect Mobile devices
    \correct All of the mentioned

  \end{choice}

\end{question}

\begin{question}
  Scheduling is one of the most important things for a radio program
  \begin{choice}
    \correct True
    \incorrect False

  \end{choice}

\end{question}

\begin{question}
  Sound effects in a radio program are distracting and should not be used
  \begin{choice}
    \incorrect True
    \correct False

  \end{choice}

\end{question}

\begin{question}
  Which file format would be best to store a radio broadcast?
  \begin{choice}
    \correct WAV
    \incorrect mp3
    \feedback better for internet because smaller size
    \incorrect ogg
    \feedback better for internet because smaller size
  \end{choice}

\end{question}

\end{document}
