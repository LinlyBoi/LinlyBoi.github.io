\documentclass[theme=sleek, randomorder, hidesidemenu]{webquiz}
\title{Regular Expressions}
\begin{document}

\begin{question}
  ``Sara has 3books, and she loves her 2cats. She finds them cute and adorable. ''\\
  Write the regular expression to get the set of all alphabetic strings.
  (Sara has books and she loves her cats She finds them cute and adorable)
  % \answer[string]{\[a-zA-Z\]+}
  \begin{choice}[columns=2]
    \incorrect \verb|[a-zA-Z]*|
    \correct \verb|[a-zA-Z]+|
    \incorrect \verb|\b[a-zA-Z]+|
    \incorrect \verb|[a-zA-Z]?|
    \incorrect \verb|[A-Z]*|
    \incorrect \verb|[A-Z]?|
  \end{choice}
\end{question}

\begin{question}
  ``Sara has 3books, and she loves her 2cats. She finds them cute and adorable. ''\\
  Write the regular expression to get the set of all words starting with a number.
  (3books 2cats)
  % \answer[string]{\texttt{\[0-9\]\[a-zA-Z\]+}}
  \begin{choice}[columns=2]
    \incorrect \verb|[0-9][a-zA-Z]|
    \incorrect \verb|[0-9][a-zA-Z]*| \feedback would match punctuation too
    \incorrect \verb|[0-9]\w| \feedback would only match the number and first letter after
    \correct \verb|[0-9][a-zA-Z]+|
  \end{choice}
\end{question}

\begin{question}
  ``Sara has 3books, and she loves her 2cats. She finds them cute and adorable. ''\\
  Write the regular expression to get the set of all alphabetic strings excluding words that start with a number.
  (Sara has and she loves her She finds them cute and adorable)
  % \answer[string]{\textbackslash{}b\[a-zA-Z\]+}
  \begin{choice}[columns=2]
    \incorrect \verb|^[0-9][a-zA-Z]+|
    \incorrect \verb|\w[a-zA-Z]+|
    \correct \verb|\b[a-zA-Z]+|
    \incorrect \verb|[a-zA-Z]+|
  \end{choice}
\end{question}

\begin{question}
  ``Sara has 3books, and she loves her 2cats. She finds them cute and adorable. ''\\
  Write the regular expression to get the set of all lowercase strings excluding words that end with e.
  % \answer[string]{\textbackslash{}b\[a-z\]+e\textbackslash{}b}
  \begin{choice}[columns=2]
    \incorrect \verb|\b[a-z]*e\b|
    \incorrect \verb|\b[a-z]*e|
    \correct \verb|\b[a-z]+e\b|
    \incorrect \verb|\b[a-z]+e|
  \end{choice}
\end{question}

\begin{question}
  ``Chanel is not pronounced channel. 3 channels are we had back when cars were cheaper''\\
  Write the regular expression to get only ``Chanel'', ``channel'' and ``channels'' ignoring case of these words (only in this sentence as context)

  % \answer{\textbackslash{}w}
  \begin{choice}[columns=2]
    \incorrect \verb|[cC]han[a-z]+|
    \feedback this would also match words like archangel
    \correct \verb|\b[cC]han[a-z]+|
    \incorrect \verb|chan[a-z]+| \feedback this lacks the capital C
    \incorrect \verb|[cC]han\w+|

  \end{choice}


\end{question}
\begin{question}
  Assume we are attempting to find the most common letter that words start with what regex expression would we use to preprocess the data?
  \begin{choice}[columns=2]
    \correct \verb|^[a-zA-Z]| \feedback this really won't in python but trust it works elsewhere :)
    \incorrect \verb|^| \feedback this will get numbers too!
    \incorrect \verb|[^a-zA-Z]| \feedback this won't match any letters!
    \incorrect \verb|^[^0-9]| \feedback this can grab symbols
  \end{choice}

\end{question}

\begin{question}
  Using the expression \verb|[tT]he| to match all instances of the word the would yield:
  \begin{choice}[columns=2]
    \correct false positives
    \incorrect false negatives
    \incorrect no matches
    \incorrect no errors

  \end{choice}

\end{question}

\begin{question}

  When negating within a regex, the caret is placed outside the square brackets.

  \begin{choice}
    \incorrect True
    \correct False
  \end{choice}

\end{question}

\begin{question}

  The consequent effect of minimising false positives.

  \begin{choice}[columns=3]
    \correct Accuracy++
    \incorrect Accuracy--
    \correct Precision++
    \incorrect Precision--
    \incorrect Recall++
    \incorrect Recal--
  \end{choice}

\end{question}

\begin{question}

  "Dial 199, for a bumpin' gud time", What would be the suitable expression to capture the hotline?\\
  (Ignoring boundaries and spaces)

  \answer[string]{\verb|[0-9]|}
  \whenRight GJ go bump yourself
  \whenWrong The numbers mason, what do they mean?? (Capture them to find out more!)
\end{question}

\begin{question}

  Say we have a list of local IP addresses, and we wanna extract them from any given text containing them.\\
  (Choose the most correct, I haven't written an absolute correct that limits the last octet to 255)

  \begin{choice}[columns=2]
    \incorrect \verb|192\.162\.1\.[0-9]| \feedback 3 Octets possible
    \correct \verb|192\.162\.1\.\d{1,3}| \feedback Yes I know, 192.162.1.999 is invalid, but it's the best I can muster rn
    \incorrect \verb|192\.162\.1[0-9]| \feedback You forgot a dot there at the end lul.
  \end{choice}

\end{question}

\end{document}
